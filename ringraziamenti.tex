\chapter*{Ringraziamenti}

Vorrei ringraziare il mio professore relatore, 
Francesco Santini, per la sua disponibilità e 
il suo supporto durante la stesura di questa tesi. 
Grazie ai miei genitori per il loro sostegno continuo e
ai miei parenti per avermi accompagnato 
in questo percorso con la loro vicinanza.

Silvia,
grazie per aver creduto in me fino in fondo. Sei stata una svolta positiva per la mia vita, era proprio quello che ci voleva.
Come sai, con le parole non sono mai stato bravo, però devi sapere che anche grazie ai tuoi incoraggiamenti sono riuscito ad arrivare dove sono oggi.
Ragazza migliore non esiste. Ti amo.
E un ringraziamento anche ai tuoi genitori.

A mio fratello Tommaso, sono felice che ultimamente ci stiamo risentendo e che tra di noi ora ci sia 
un bel rapporto. Spero davvero che possa continuare così anche in futuro.

Partendo dal mio amico di più vecchia data, er Nicco:
ti ringrazio per esserci stato fin da subito. Sono felice di avere un’amicizia così duratura; sei stato una sorta di guida.
Una delle tue qualità migliori, che ho sempre apprezzato e che poi ho cercato di fare mia, è sempre stata la tua shallezza e il tuo umorismo.

Passando al Dazzo, posso dire che, a parte il tuo cronico ritardo,
mi hai sempre fatto ridere e mi sono sempre sentito a mio agio con te.
Non mi scorderò mai quelle girate a Casaglia, dove cercavi di spaventarmi con le storie sulle messe nere. Epico.

Arrivando alla zona lacustre...
Permettimi di dire che non ho sentito la puzza di lago, ma solo l’odore dell’affetto di cui avevo bisogno.
Un’aria nuova che prima non avevo mai sperimentato.
Anche se era il 2020, un anno un po’ oscuro, per me è stato uno dei momenti più belli della mia vita.
In generale, avete avuto tutti un impatto sulla mia vita che nemmeno vi rendete conto. Grazie.

Racheleee,
da quando ti ho conosciuta ti ho sempre vista come una persona solare, e ti ringrazio per avermi sempre ascoltato.
Sei stata disponibilissima fin dall’inizio, e lo apprezzo tantissimo.
La cosa più bella è che a entrambi piacciono i dolci… viva i kaiserrrr!

Il proibito, aka Sabba:
grazie per i tuoi preziosi suggerimenti, che ancora tengo a mente.
Avevi ragione su molte cose, tra cui quella di fregarmene del giudizio degli altri.
Sono state bellissime le serate passate a vedere quei film insieme: davvero indimenticabili.

Lusciaa, aka Lux:
ti ringrazio di essere mia amica davvero, sei una grande.
La tua è sempre stata un’amicizia sincera, e spero che potremo rimanere amici come siamo adesso.
Abbiamo fatto chiacchierate di ogni tipo, e tu sei sempre stata lì ad ascoltarmi.
Penso che con te ho raggiunto il livello massimo d’intesa possibile: è incredibile come riusciamo a divertirci con qualsiasi cavolata ci passi per la testa.
Mi hai insegnato molte cose in questi anni, e anche se non lo sai, a volte sai essere molto saggia.
Sei come un fratello per me, grazie.

La Francesca sbagliata, aka la Noce:
anche se ci sono stati dei brutti trascorsi, sono contento che ora abbiamo un bellissimo rapporto.
Ti ringrazio per l’aiuto con lo studio e con i teoremi.
Anche se quell’esame è stato terribile, in un certo senso mi ha fatto capire che sei una persona davvero disponibile.

La Francesca giusta, aka la Sorbin:
grazie di tutto, sei sempre stata gentile e disponibile.
All’inizio, quando vi ho conosciuto, ero parecchio timido, ma con te ho capito subito che potevo parlare di tutto senza sentirmi giudicato.

Sha e le sue rotonde:
mi hai sempre fatto ridere. Hai un bellissimo senso dell’umorismo e, in molti casi, grazie alla tua autorità di bionda, siamo riusciti a organizzare vacanze fantastiche… incluso lo scorso capodanno.

Un ringraziemento speciale va ad Alberto, alle tutte le zie 
(Claudia, Nadia, Paola, Antonella),
a Caterina e Costanza,
alle mie due nonne, Vitto, Stefano, Marcella,
Riccardo, Sara, Jessica, Giuseppe La Capra, Pietro, Romans, Saverio, Kevin e Barbara. 

Un pensiero particolare alla mia cugina Francesca e al mio fratello Tommaso.


\vspace{17mm}

\vs
\begin{flushright}
 Federico
\end{flushright}

