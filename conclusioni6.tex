\chapter{Conclusione}
Questa tesi ha analizzato in dettaglio il Data Distribution Service (DDS),
iniziando con un'introduzione in cui viene spiegato il suo funzionamento 
e le sue caratteristiche principali, incluse l'interoperabilità tra 
le diverse implementazioni tra diversi vendors, la possibilità di 
cambiare la topologia della rete a runtime e una gestione di policy 
QoS che rendono il middleware estremamente flessibile per ogni tipo 
di utilizzo. 

Nei successivi capitoli, dopo l'introduzione, sono state 
analizzate 
delle vulnerabilità relative allo stesso standard OMG che il 
DDS deve rispettare. Inoltre sono state anche mostrate delle
possibili soluzione per mitigare certe problematiche di
sicurezza, tra cui il DDS security che tramite i suoi plugin 
permette di crittare le comunicazioni tra le varie entità e 
bloccare certe loro azioni se non hanno determinati permessi.

Un'altra tipologia di vulnerabilità che é stata presentata é 
relativa al software DDS utilizzato dai vendors per 
implementarlo. Queste vulnerabilità hanno evidenziato che 
soluzioni open source come FastDDS hanno maggiori possibilità 
di avere problematiche relative alla sicurezza dato che molti 
la maggior parte dei CVE compilati sono relativi a questa 
implementazione di eProsima. Da qui possiamo dedurre che 
anche se il costo iniziale di software DDS a pagamento 
é maggiore garantiscono una maggiore sicurezza.

Queste vulnerabilità relative all'implementazione hanno anche
evidenziato 
come lo standard OMG puó creare diverse problematiche 
legate alla sicurezza. 
Anche se quest'ultime ormai sono note, molte volte vengono ignorate 
dato che per i vendors l'importante é mantenere il loro applicativo 
in regola con lo standard OMG. Questa scelta é guidata dal fatto che 
lo standard garantisce l'interoperabilità tra i diversi vendor. 
Purtroppo la collaborazine di queste software house sono ancora 
molto limitate tra di loro, se in futuro ci fosse piú collaborazione 
si potrebbe arrivare a creare standard piú rigidi in modo 
da aumentare la sicurezza complessiva del middleware.

OMG in futuro dovrebbe aggiornare i propri standard in modo da 
mitigare queste problematiche legate alle sue regole di interoperabilità, 
non solo utilizzando il DDS security, specialmente durante il 
processo di discovery. Uno standard che OMG ha proposto 
é il DDS security che peró 
non viene sempre applicato dato che in molti casi é difficile da 
configurare in una rete giá esistente e sprovvista di questa estensione.

Una soluzione potrebbe essere quella di aumentare solo la sicurezza del 
processo di discovery che in molti casi é il 
principale vettore d'attacco, magari 
utilizzando una parte dei plugin di sicurezza giá impiegati dal DDS security.

Nell'ultima parte dell'elaborato sono stati descritti degli strumenti 
in grado di analizzare il traffico all'interno di una rete DDS.
É stato introdotto WireShark un potente strumento in grado di 
catturare pacchetti del protocollo RTPS insiema ad una sua alternativa
sviluppata da eProsima chiamata eProsima DDS Record \&
Replay. Mentre questi due strumenti sono unici e non sono stati 
sviluppati piú volte da differenti software house, lo stesso non 
si puó dire per il tool Fast DDS Spy che é stato sviluppato 
con diversi nomi per quasi tutte le implementazioni DDS. 
Anche se questi strumenti garantiscono il funzionamento 
con diversi applicativi DDS sviluppati da varie software house, 
presentano delle differenze e punti di forza differenti. 

Se OMG e i diversi vendors in futuro collaborasseró tra di loro 
potrebbero creare un applicativo definitivo per monitorare 
una rete DDS avendo tutti i vantaggi possibili. Questo applicativo 
evita di creare differenze di analisi del traffico che normalmente 
si possono trovare quando vengono adoperati i tool a oggi 
disponibili.
Un altro sistema aggiuntivo per migliorare l'efficacia del 
controllo del traffico potrebbe essere quello 
dell'utilizzo di un'intelligenza artificiale che monitora 
i pacchetti scambiati tra le varie entità. 
Ad esempio essa puó monitorare se i dati ricevuti dai sensori 
rimangono all'interno di un intervallo predefinito e se 
mantentengono una consistenza nel tempo. Se viene rilevato 
un problema, successivamente puó avvisare un operatore che 
analizzerà piú nel dettaglio l'inconsistenza indivduata 
dall'AI

Inoltre questa AI potrebbe 
essere impiegata durante la configurazione iniziale della rete,
specialmente durante la creazione delle policy QoS per evitare
di creare delle misconfigurazioni che possono essere
successivamente sfruttate da un attore malevolo.


