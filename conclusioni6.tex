\chapter{Conclusione}
Questa tesi ha analizzato in dettaglio il Data Distribution Service (DDS),
iniziando con un'introduzione in cui viene spiegato il suo funzionamento 
e le sue caratteristiche principali, incluse l'interoperabilità tra 
le diverse implementazioni tra diversi vendors, la possibilità di 
cambiare la topologia della rete a runtime e una gestione di policy 
QoS che rendono il middleware estremamente flessibile per ogni tipo 
di utilizzo. 

Nei successivi capitoli, dopo l'introduzione, sono state 
analizzate 
delle vulnerabilità relative allo stesso standard OMG che il 
DDS deve rispettare. Inoltre sono state anche mostrate delle
possibili soluzioni per mitigare certe problematiche di
sicurezza, tra cui il DDS security che tramite i suoi plugin 
permette di crittare le comunicazioni tra le varie entità e 
bloccare certe loro azioni se non hanno determinati permessi.

Un'altra tipologia di vulnerabilità che é stata presentata é 
relativa al software DDS utilizzato dai vendors per 
implementarlo. Esse hanno evidenziato che 
soluzioni open source come FastDDS hanno maggiori possibilità 
di avere problematiche relative alla sicurezza dato che
la maggior parte dei CVE compilati sono relativi a questa 
implementazione di eProsima. Da qui possiamo dedurre che 
anche se il costo iniziale di un software DDS a pagamento 
é maggiore, alla lunga garantiscono una maggiore sicurezza.

Queste vulnerabilità relative all'implementazione hanno anche
evidenziato 
come lo standard OMG puó creare diverse problematiche 
legate alla sicurezza. 
Anche se quest'ultime molte delle vulnerabilitá che sono state presentate 
sono ormai note, in molti casi vengono ignorate 
dato che i vendors cercano di mantenere il loro applicativo 
in regola con lo standard OMG. Questa scelta é guidata dal fatto che 
lo standard garantisce l'interoperabilità tra le diverse implementazioni. 
Purtroppo la collaborazine di queste software house sono ancora 
molto limitate tra di loro, se in futuro ci fosse piú collaborazione 
si potrebbe arrivare a creare standard piú rigidi in modo tale
da aumentare la sicurezza complessiva del middleware.

OMG in futuro dovrebbe aggiornare i propri standard in modo da 
mitigare queste problematiche legate a queste regole di interoperabilità
specialmente durante il 
processo di discovery. Lo standard OMG proposto per rispondere alla problematica
é il DDS security che purtroppo in molti casi 
non viene applicato dato che é difficile da 
configurare in una rete giá esistente e sprovvista di questa estensione.

Una soluzione potrebbe essere quella di aumentare solo la sicurezza del 
processo di discovery che in molti casi é il 
principale vettore d'attacco, magari 
utilizzando una parte dei plugin di sicurezza giá impiegati dal DDS security.

Nell'ultima parte dell'elaborato sono stati descritti degli strumenti 
in grado di analizzare il traffico all'interno di una rete DDS.
É stato introdotto WireShark un potente strumento in grado di 
catturare pacchetti del protocollo RTPS insieme ad una alternativa
sviluppata da eProsima chiamata eProsima DDS Record \&
Replay. Mentre questi due strumenti sono unici e non sono stati 
sviluppati da differenti software house, lo stesso non 
si puó dire per il tool Fast DDS Spy che é stato sviluppato 
piú volte con diversi per ogni possibile 
applicativo DDS.
Anche se questi strumenti garantiscono il funzionamento 
con piú applicativi del middleware creati dai vari vendors, essi
presentano dei distinti punti di forza. 

Se OMG e i diversi vendors in futuro collaborasseró tra di loro 
potrebbero creare un applicativo definitivo per monitorare 
una rete DDS avendo tutti i vantaggi possibili. Questo applicativo 
eviterebbe di creare differenze di analisi del
traffico che normalmente 
si possono trovare quando vengono adoperati i tool a oggi 
disponibili.
Un altro sistema aggiuntivo per migliorare l'efficacia del 
controllo del network potrebbe essere quello 
dell'utilizzo di un'intelligenza artificiale che monitora 
i pacchetti scambiati tra le varie entità. 
Ad esempio essa potrebbe monitorare se i valori ricevuti dai sensori 
rimangono all'interno di un intervallo predefinito e se 
mantentengono una consistenza nel tempo. Se viene rilevato 
un problema, l'IA puó avvisare un operatore che 
analizzerà piú nel dettaglio l'inconsistenza individuata.

Inoltre questa AI potrebbe 
essere impiegata durante la configurazione iniziale della rete,
specialmente durante la creazione delle policy QoS per evitare
di creare delle misconfigurazioni che possono essere
successivamente sfruttate da un attore malevolo.

L'ultimo strumento che é stato presentato all'interno di questo 
elaborato é DDSFuzz, un software che ci permette di effettuare test 
di tipo fuzzing per ogni implementazione del DDS. Infatti una delle 
vulnerabilità che é stata presentata é stata scoperta proprio 
con l'utilizzo di un tool simile. Questo ci dimostra che questi 
modelli di software sono molti utili per trovare falle di sicurezza 
che possono compromettere anche il funzionamento dei dispositivi
all'interno della rete. DDSFuzz é stato descritto in maniera dettagliata
perché rappresenta il primo applicativo di tipo fuzzer ad essere
solamente  
compatibile con il middleware DDS portando l'analisi dei suoi test di 
input a una maggiore accuratezza. Come é stato indicato, i bug identificati
si divono in due categorie distinte, la prima relativa ai tradizionali 
bug, ma la seconda, il vero punto di forza di DDSFuzz sono i bug semantici.
Solo grazie all'esecuzione parallela di diversi applicativi DDS é possibile
individuare quest'ultima categoria di bug che avvengono quando 
una o piú delle norme del DDS, come ad esempio l'obbligo di 
autenticazione puó essere bypassato con uno specifico input.

DDSFuzz é relativamente nuovo e ci dimostra una che é presente 
una mancanza di analisi per trovare problematiche di funzionamento
del DDS e alla sua sicurezza, dato che sono subito state trovate 
nuove CVE dopo la sua esecuzione. 



