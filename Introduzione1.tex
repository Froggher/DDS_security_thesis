\chapter{Introduzione al DDS}
In questo capitolo viene fatta un'introduzione generale dello 
standard del Data Distribution Service (DDS) gestita
dall'Object Management Group (OMG). Inizialmente a livello generale 
per poi andare sempre più nel
dettaglio per capire il suo funzionamento, che ci sarà utile 
capire per comprendere le vulnerabilità che verranno analizzate in
successivi capitoli. Inoltre verrà introdotta la sua estensione
DDS security che si occupa di rendere più sicuro lo standard DDS 
aggiungendo elementi di sicurezza come l'autenticazione e una 
implementazione della cifratura dei pacchetti scambiati tra i vari
dispositivi connessi alla rete. Successivamente verranno mostrate delle 
implementazioni e in quali contesti viene utilizzato utilizzato
attualmente e in futuro. 
Infine verra mostrato in quali contesti attuali e futuri viene
utilizzato con le sue diverse implementazioni.

In questa tesi, in dei casi, verrà omessa la specifica OMG perchè 
ci riferiremo
esclusivamente al DDS conforme all'Object Management Group. Questo 
standard ha delle specifiche tecniche ben precise, consente
l'interoperabilità tra i diversi vendor che lo rispettano, insieme
a tanti altri numerosi vantaggi.
\cite{dds1.4}


\section{Modello publish/subscribe}
Prima di paralare del DDS, bisogna prima capire il funzionamento
del modello Data-Centric Publish-Subscribe (DCPS) 
che sta alla base di tutto il suo funzionamento.
Questo sistema di publish e subscribe non funziona come la 
classica applicazione che siamo abituati a vedere tra server e
client nell'ambito delle comunicazioni o anche rispetto a 
modelli Peer-To-Peer.
Nel nostro caso avremo due entitá principali
che possono comunicare tra di loro quando hanno un topic, che
rappresentano una tipologia di dati, (ad esempio temperatura, 
distanza, velocitá, etc...) in comune tra di loro.
\begin{itemize}
    \item Publisher: colui che "pubblica" nuovi dati riguardanti dei
    topic rendendoli accessibili ai subscriber iscritti. 
    Di solito si tratta di un sensore.
    \item Subscriber: colui che si "iscrive" ai topic del publisher, 
    cominciando
    cosí a ricevere nuovi dati sul topic scelto. Molte volte si tratta
    di un dispositivo utilizzato per mostrare informazioni, come un
    semplice schermo.
\end{itemize}
Una caratteristica di queste entitá é che possono essere aggiunte o rimosse
senza nessun problema, sia publisher che subscriber, dato che le 
comunicazioni avvengono in modalitá asincrona. Un publisher come impostazione
predefinita non deve ricevere conferma di ricezione da parte del 
subscriber a cui manda i pacchetti contenenti i dati riguardo il topic.
In questo modo il publisher puó continuamente mandare nuovi dati
senza effettuare operazioni di conferma, rendendendo cosí le comunicazioni
molto piú responsive. 

Un'altra qualitá del modello
publish/subscribe si manifesta quando
un publisher deve inviare dei nuovi dati a piú subscriber. 
In tale situazione, un
modello multicast viene utilizzato per i mandare i pacchetti, che vengono 
poi ricevuti da un'entitá che si occuperá di mandare ai subscriber
le nuove informazioni riguardanti i topic a cui sono iscritti.
Il modello cosí risulta molto
flessibile e utilizzabile in ambienti real-time dove le fonti delle
comunicazioni possono cambiare o essere utilizzate da piú dispostivi,
ad esempio i subscriber possono avere cambiare i topic a cui 
sono iscritti e/o i publisher possono smettere di pubblicare nuove
informazioni su un determinato topic.\cite{OH2010318}



\section{Modello DDS}
Sto facendo un test di prova riguardante questo testo scritto
tipo lorem ipsum lorem ipsum lorem ipsum lorem ipsum lorem ipsum
lorem ipsum lorem ipsum lorem ipsum lorem ipsum lorem ipsum lorem ipsum
lorem ipsum lorem ipsum lorem ipsum lorem ipsum lorem ipsum lorem ipsum
lorem ipsum lorem ipsum lorem ipsum lorem ipsum lorem ipsum lorem ipsum
lorem ipsum lorem ipsum lorem ipsum lorem ipsum lorem ipsum lorem ipsum
lorem ipsum lorem ipsum lorem ipsum lorem ipsum lorem ipsum lorem ipsum
lorem ipsum lorem ipsum lorem ipsum lorem ipsum lorem ipsum lorem ipsum
lorem ipsum lorem ipsum lorem ipsum lorem ipsum lorem ipsum lorem ipsum
lorem ipsum lorem ipsum lorem ipsum lorem ipsum lorem ipsum lorem ipsum