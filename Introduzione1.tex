\chapter{Introduction}


In questo capitolo ci occuperemo di analizzare e comprendere delle vulnerabilità
del protocollo DDS standard OMG (Object Managment Group). In particolare
verrà analizzato il vettore d'attacco, il protocollo utilizzato, il bersaglio
dell'attacco e infine verrà proposta una soluzione applicabile per
mitigare possibili attacchi non autorizzati. Successivamente grazie all'aiuto 
di un software riusciremo a capire come queste vulnerabilità possono 
influire sul funziomanto degli --host-- collegati alla rete DDS.

\section{Attacchi di tipo DDoS}
Questo attacco consiste nel sovraccaricare un dispositivo collegato alla rete DDS
in modo tale da renderlo inutilizzabile. Infatti dato che i dispositivi collegati
sono di -- tipo I O T -- la potenza di calcolo nella maggior parte dei casi sarà
molto ridotta. Inoltre in molti casi ci possiamo ritrovare ad utilizzare dispositivi
che non possono permettersi --delay-- nell'analisi di certi dati, specialmente in
ambiti dove bisogna avere una risposta sempre rapida e disponibile, come ad esempio
nel campo della medicina e nel campo militare.



\subsection{Sottosezione}


% Definizione di un colore personalizzato
\definecolor{customgray}{rgb}{0.70, 0.70, 0.70} % Grigio chiaro

% Regolazione dello spessore delle linee
\setlength{\arrayrulewidth}{1.0pt} % Spessore linee generali
% \renewcommand{\arraystretch}{1.2} % Altezza righe


\begin{table}[H]
    \centering
    \rowcolors{2}{black!5}{white}
    \resizebox{\linewidth}{!}{%
        \begin{tabular}{|c|c|c|c|c|c|}
            \hline
            \rowcolor{customgray}
            \multicolumn{1}{|>{\columncolor{customgray}}c|}{\tabularCenterstack{c}{\textbf{Tipo di}\\ \textbf{attacco}}} &
            \multicolumn{1}{>{\columncolor{customgray}}c|}{\tabularCenterstack{c}{\textbf{Vettore} \\ \textbf{attacco}}} &
            \multicolumn{1}{>{\columncolor{customgray}}c|}{\tabularCenterstack{c}{\textbf{Protoc.}/ \\ \textbf{Estens.}}} &
            \multicolumn{1}{>{\columncolor{customgray}}c|}{\tabularCenterstack{c}{\textbf{Bersaglio} \\ \textbf{nella rete}}} &
            \multicolumn{1}{>{\columncolor{customgray}}c|}{\tabularCenterstack{c}{\textbf{Software}}} &
            \multicolumn{1}{>{\columncolor{customgray}}c|}{\tabularCenterstack{c}{\textbf{Soluzione}}} \\
            \hline
            \tabularCenterstack{l}{Discovery \\ devices[2]} &
            \tabularCenterstack{c}{Verbose nature \\ of RTPS} &
            \tabularCenterstack{c}{DDSI-RTPS} &
            \tabularCenterstack{c}{Tutti i par-\\tecipanti} &
            \tabularCenterstack{c}{Sniffer \\ python} &
            \tabularCenterstack{c}{DDS-security} \\
            \specialrule{0.3pt}{0pt}{0pt} % Linea più spessa dopo l'intestazione
            \tabularCenterstack{l}{DDos[2]} &
            \tabularCenterstack{c}{Heartbeat \\ sequence number} &
            \tabularCenterstack{c}{DDSI-RTPS} &
            \tabularCenterstack{c}{Data-reader} &
            \tabularCenterstack{c}{Sniffer \\ python} &
            \tabularCenterstack{c}{DDS-security} \\
            \specialrule{0.3pt}{0pt}{0pt} % Linea più spessa dopo l'intestazione
            \tabularCenterstack{l}{DDoS[3]} &
            \tabularCenterstack{c}{Authentication \\ challenge} &
            \tabularCenterstack{c}{DDS security 1.1 \\ Discovery protoc.} &
            \tabularCenterstack{c}{Tutti i par-\\tecipanti} &
            \tabularCenterstack{c}{Proverif} &
            \tabularCenterstack{c}{Scandenza richieste \\ di autenticazione} \\
            \specialrule{0.3pt}{0pt}{0pt} % Linea più spessa dopo l'intestazione
            

            \hline
        \end{tabular}
        }
        \caption{La versione DDS in tutti i casi è la 1.4}
    \end{table}




