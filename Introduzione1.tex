\chapter{Introduction}


In questo capitolo ci occuperemo di analizzare e comprendere delle vulnerabilità
del protocollo DDS standard OMG (Object Managment Group). In particolare
verrà analizzato il vettore d'attacco, il protocollo utilizzato, il bersaglio
dell'attacco e infine verrà proposta una soluzione applicabile per
mitigare possibili attacchi non autorizzati. Successivamente grazie all'aiuto 
di un software riusciremo a capire come queste vulnerabilità possono 
influire sul funziomanto degli --host-- collegati alla rete DDS.


\section{Attacchi di tipo DDoS}
Questo attacco consiste nel sovraccaricare un dispositivo collegato alla rete DDS
in modo tale da renderlo inutilizzabile. Infatti dato che i dispositivi collegati
sono di -- tipo I O T -- la potenza di calcolo nella maggior parte dei casi sarà
molto ridotta. Inoltre in molti casi ci possiamo ritrovare ad utilizzare dispositivi
che non possono permettersi --delay-- nell'analisi di certi dati, specialmente in
ambiti dove bisogna avere una risposta sempre rapida e disponibile, come ad esempio
nel campo della medicina e nel campo militare.


\subsection{DDoS blocco ricezione da parte del data-reader Foglio 2}

%\subsubsection{Prologo}
Citazioni da foglio 2 a gogo
Il vettore di attacco si trova nell'implementazione del DDS chiamata DDSI-RTPS che
si occupa di scambiare messaggi tra i data-reader (coloro che si iscrivono ai vari
ai vari topic) e i data-subscribers (di solito sono sensori che mandano dati). Per
comunicare questi dispositivi utilizzano il protocollo RTPS. 
Questo protocollo utilizza il messaggio HEARTBEAT che viene mandato da un data-writer 
a un data-reader per specificare il sequence number nel data-writer.
All'interno del messaggio HEARTBEAT troviamo il sequence number che serve al 
data-reader per sincronizzarsi con il data-writer durante la ricezione dei messaggi. 
Infatti il data-reader quando riceve il sequence number all'interno di un HEARTBEAT
può identificare se ci sono dei pacchetti mancanti e segnalarli al
data-writer.

Un data-writer inoltre può richiedere un messaggio ACKNACK da un data-reader se 
nel messaggio HEARTBEAT inviato dal data-writer viene specificata la flag FINAL.
In casi in cui bisogna essere certi che il data-reader riceve tutti i dati del
data-writer, quest'ultimo manda un HEARTBEAT con la flag FINAL impostata, al
data-reader che successivamente deve rispondere necessariamente con un messaggio
ACKNACK per confermare la ricezione nel messaggio HEARTBEAT.
I controlli HEARTBEAT effettuati dal data-reader,
infatti non sono sufficienti a coprire questo tipo di attacco
dato che quest'ultimo:
\begin{itemize}
    \item esegue un check per verificare che non vi siano numeri negativi
    \item controlla che l'ultimo sequence number arrivato non ha un valore più alto 
    del sequence number ricevuto in precedenza 
\end{itemize}
\subsubsection{Dettagli attacco}
Per sfruttare questa vulnerabilità l'attaccante deve utilizzare qualche strumento
per sniffare la comunicazione tra data-reader e data-writer e intercettare un
messaggio di tipo HEARTBEAT. Successivamente l'attaccante modifica il valore del
sequence number del messaggio HEARTBEAT. Il messaggio poi viene mandato verso il
data-writer così facendolo rimanere in attesa di un messaggio HEARTBEAT con Un
sequence number superiore a quello appena ricevuto. Facendo così il data-reader
non legge più i messaggi mandati dal data-writer e bloccando il così
il funziomanto del data-reader finchè il sequence number non
sarà superiore a quello ricevuto dall'attaccante.


\subsubsection{Questa è una sottosottosezione}
La teoria dell'attacco ci dice che se 

% Definizione di un colore personalizzato
\definecolor{customgray}{rgb}{0.70, 0.70, 0.70} % Grigio chiaro

% Regolazione dello spessore delle linee
\setlength{\arrayrulewidth}{1.0pt} % Spessore linee generali
% \renewcommand{\arraystretch}{1.2} % Altezza righe


\begin{table}[H]
    \centering
    \rowcolors{2}{black!5}{white}
    \resizebox{\linewidth}{!}{%
        \begin{tabular}{|c|c|c|c|c|c|}
            \hline
            \rowcolor{customgray}
            \multicolumn{1}{|>{\columncolor{customgray}}c|}{\tabularCenterstack{c}{\textbf{Tipo di}\\ \textbf{attacco}}} &
            \multicolumn{1}{>{\columncolor{customgray}}c|}{\tabularCenterstack{c}{\textbf{Vettore} \\ \textbf{attacco}}} &
            \multicolumn{1}{>{\columncolor{customgray}}c|}{\tabularCenterstack{c}{\textbf{Protoc.}/ \\ \textbf{Estens.}}} &
            \multicolumn{1}{>{\columncolor{customgray}}c|}{\tabularCenterstack{c}{\textbf{Bersaglio} \\ \textbf{nella rete}}} &
            \multicolumn{1}{>{\columncolor{customgray}}c|}{\tabularCenterstack{c}{\textbf{Software}}} &
            \multicolumn{1}{>{\columncolor{customgray}}c|}{\tabularCenterstack{c}{\textbf{Soluzione}}} \\
            \hline
            \tabularCenterstack{l}{Discovery \\ devices[2]} &
            \tabularCenterstack{c}{Verbose nature \\ of RTPS} &
            \tabularCenterstack{c}{DDSI-RTPS} &
            \tabularCenterstack{c}{Tutti i par-\\tecipanti} &
            \tabularCenterstack{c}{Sniffer \\ python} &
            \tabularCenterstack{c}{-} \\
            \specialrule{0.3pt}{0pt}{0pt} % Linea più spessa dopo l'intestazione
            \tabularCenterstack{l}{DDos[2]} &
            \tabularCenterstack{c}{Heartbeat \\ sequence number} &
            \tabularCenterstack{c}{DDSI-RTPS} &
            \tabularCenterstack{c}{Data-reader} &
            \tabularCenterstack{c}{Sniffer \\ python} &
            \tabularCenterstack{c}{-} \\
            \specialrule{0.3pt}{0pt}{0pt} % Linea più spessa dopo l'intestazione
            \tabularCenterstack{l}{DDoS[3]} &
            \tabularCenterstack{c}{Authentication \\ challenge} &
            \tabularCenterstack{c}{DDS security 1.1 \\ Discovery protoc.} &
            \tabularCenterstack{c}{Tutti i par-\\tecipanti} &
            \tabularCenterstack{c}{Proverif} &
            \tabularCenterstack{c}{Scandenza richieste \\ di autenticazione} \\
            \specialrule{0.3pt}{0pt}{0pt} % Linea più spessa dopo l'intestazione
            
            % Aggiungere altre linee

            \hline
        \end{tabular}
        }
        \caption{La versione DDS in tutti i casi è la 1.4}
    \end{table}




