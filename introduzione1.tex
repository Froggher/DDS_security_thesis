\chapter{Introduzione}
In questo elaborato verrà esposto il middleware Data Distribution 
Service (DDS) standardizzato dall'Object Management Group (OMG) e 
verrà effettuata un'analisi della sua sicurezza.
In particolare nel Capitolo~\ref{introdDDS} 
viene spiegato inizialmente 
il suo funzionamento, per poi passare ai successivi due capitoli 
dove verranno introdotte due tipologie di vulnerabilità: una 
basata sullo standard OMG, mentre l'altra si basa sugli applicativi
che utilizzano il DDS. Infine nel Capitolo~\ref{chtools} 
verranno 
mostrati dei tools in grado di analizzare il analizzare 
il traffico di una rete DDS e inoltre verrà introdotto DDSFuzz, 
un tool per eseguire test di tipo fuzz.

Il DDS di OMG viene adoperato in numerosi campi grazie alla 
sua comunicazione efficiente che può essere impostata anche 
per funzionare in sistemi che hanno bisogno di risposte 
real-time. L'architettura è basata su un sistema di comunicazione 
distribuito che si basa su modello publish/subscribe in modo 
tale da poter rendere flessibile la topologia della rete.
Questo vantaggio porta il DDS ad essere utilizzato in vari campi 
che hanno necessità di aggiornare continuamente la loro 
struttura. Viene adottato in medicina, in ambienti militari, 
in controlli industriali, in veicoli autonomi e nelle telecomunicazioni.

Un'applicazione molto importante del DDS è ROS Robot Operating System
uno strumento che permette a molti ingegneri robotici di 
interfacciarsi con vari dispositivi all'interno di una 
rete che hanno bisogno di effettuare uno scambio di informazioni.
Ad esempio un utente può ricevere dati da diversi robot e sensori
contemporaneamente, inclusi 
movimento dei giunti (espressi in coordinate), velocità, 
posizione geografica, stato del robot, temperatura di un 
determinato componente e tutti gli altri sensori di cui il 
robot è dotato. Il DDS all'interno di ROS è utilizzato come 
middleware e si occupa di stabilire e mostrare i dati 
che vengono mandati dai componenti della rete. 
In questo modo l'utente ha la possibilità di visualizzare 
comodamente in unico punto tutte le informazioni necessarie 
senza dover creare nuovi canali di comunicazione.

Tuttavia, nonostante i numerosi il DDS presenta alcune 
problematiche a livello di sicurezza che possono 
essere sfruttate da un attore malevolo. Una delle 
caratteristiche, chiamato processo di discovery,
più utilizzate è la possibilità permettere 
a nuovi dispositivi di connettersi ad una rete DDS senza 
l'intervento di un operatore. Diverse vulnerabilità che verranno 
analizzate successivamente all'interno dell'elaborato, nel 
Capitolo~\ref{chvulstdndds} e nel Capitolo~\ref{chvulimpldds}, 
mostrano come il processo di discovery, se non adeguatamente 
configurato da i vendors delle implementazioni DDS, posso 
essere sfruttato come vettore di attacco. Proprio durante 
l'esecuzione di questo processo è possibile immettere nella 
rete pacchetti malevoli che possono compromettere il 
corretto funzionamento di uno o più dispositivi all'interno 
della rete. Inoltre, un attaccante in ascolto 
che riesce a catturare 
il traffico generato da questo processo è in grado 
di effettuare una ricognizione della rete in modo da 
ricevere abbastanza metadati, tra cui il nome del topic,
per comprendere al meglio 
il funzionamento di ogni singolo dispositivo collegato.

Un'altra caratteristica vantaggiosa per il DDS che può 
essere adoperata da un attore malevolo sono le policy QoS.
Esse servono per impostare diversi criteri di
comportamento che i dispositivi all'interno della rete 
devono seguire, inclusa la priorità di pubblicazione.
Se un attore malevolo riesce ad avere accesso ai valori 
di questi parametri esso può ottenere un controllo sui 
vari flussi dati e effettuare una manipolazione 
dei dati trasmessi che può essere 
successivamente reindirizzata attraverso una policy modificata 
dall'attaccante che 
si occupa di gestire le 
priorità verso dispositivi di ricezione. Essa può compromettere
un o più delle loro operazioni che utilizzano 
i dati ricevuti per effettuare un qualche tipo di elaborazione.

OMG per mitigare queste problematiche relative alla sicurezza 
ha creato un nuovo standard chiamato DDS security che viene
gestito come una estensione per il DDS. Il DDS security introduce 
una serie di plugin relativi all'autenticazione, crittografia e 
gestione degli accessi. Questi plugin vengono adoperati come misure 
di protezione dei messaggi scambiati per garantirne l'integrità 
l'impossibilita di effettuare manipolazioni indesiderate. 
Il processo di autenticazione effettuato dal DDS security 
ha il compito di garantire che non ci siano partecipanti della 
rete non autorizzati impedendo ad un attaccante di leggere 
il contenuto del loro traffico. Invece, la gestione degli accessi
si occupa di certificare che i dispositivi all'interno del network
hanno i permessi necessari per compiere determinate azioni.

Il DDS security in questo elaborato viene proposte diverse volte 
come la soluzione ottimale per la risoluzione di vulnerabilità. 
Tuttavia, non viene ancora impiegato in molte implementazioni, 
data la difficoltà di configurazione iniziale e l'aumento 
delle risorse richieste durante le operazione crittografiche. 
Esse risultano molto difficoltose specialmente 
per dispositivi di tipo IoT (internet of things).

In altri casi il DDS security non viene adoperato perché il 
network della rete DDS si trova all'interno di un sistema 
chiuso, isolato da internet o con un accesso limitato 
(dietro un firewall). Un esempio di questa configurazione 
si può trovare all'interno di fabbrica che collega i vari 
macchinari in modo tale da poter comunicare tra di loro e 
che non devono essere disponibili al di fuori di 
quell'ambiente.

Nel Capitolo~\ref{introdDDS} di questo elaborato si introdurrà il 
funzionamento del middleware DDS partendo dal funzionamento 
di un modello di tipo publish/subscribe confrontandolo 
con il più comune protocollo TCP/IP. Il DDS viene definito 
data-centric, dando importanza ai dati stessi semplificando 
la sua configurazione. Verrà inoltre introdotto il concetto 
delle policy QoS e del Global Space Data per permettere 
alle entità della rete di comunicare tra di loro senza preoccuparsi 
dell'accessibilità e disponibilità dei dati. Si analizzerà la sua
struttura, tutte le sue entità utilizzate introducendo il protocollo 
RTPS (Real-Time Publish-Subscribe Protocol) e i moduli che lo compongono.
Successivamente verrà introdotto il DDS security insieme ai suoi 
plugin e al loro funzionamento inclusi l'Authentication Service Plugin e 
l'Access Control Service Plugin, mostrando nel dettaglio come 
avviene il processo di autenticazione e di controllo permessi.
Infine verranno presentate diverse implementazioni DDS dei vendor, 
confrontando tra di loro soluzioni gratuite e a pagamento.

Nel Capitolo~\ref{chvulstdndds} ci occuperemo di analizzare 
le vulnerabilità del DDS che si trovano nello standard OMG. 
Queste non vengono presentate tramite una implementazione 
a differenza del Capitolo~\ref{chvulimpldds}, ma solo attraverso 
l'utilizzo dello standard OMG per evidenziare delle problematiche 
di sicurezza più generali del middleware. Le prime due vulnerabilità 
impiegano entrambe il protocollo RTPS per essere applicate, 
in particolare nel primo caso verrà introdotta una vulnerabilità 
che permette ad un attore malevolo di compiere una ricognizione 
della rete, mentre nel secondo caso verrà mostrato come la modifica del 
valore sequence number all'interno di un pacchetto RTPS può
compromettere
il funzionamento di un DataReader. 
Nel successivo attacco dimostreremo che si può creare un vettori 
di attacco anche impiegando il DDS security, in particolare durante
la fase di autenticazione di un'entità all'interno della rete.
Infine le ultime due vulnerabilità che verranno spiegate, si basano sulla 
misconfigurazione di due policy QoS in modo tale da consentire ad un
attaccante di manipolare gli aggiornamenti di un flusso di dati e 
impedire ad una o più entità di ricevere nuove informazioni riguardanti 
un topic.

Nel Capitolo~\ref{chvulstdndds} verranno introdotte altre vulnerabilità,
questa volta basati sulle implementazioni dei vari vendors DDS. 
Si analizzeranno tre casi che saranno relativi a ROS che impiegherà 
il DDS come middleware. Inizialmente viene spiegata una libreria, che
successivamente verrà fusa con il progetto Scapy di Python, che facilità
la costruzione di pacchetti RTPS. La prima vulnerabilità che verrà 
mostrata consente di effettuare una ricognizione della rete DDS 
mandando un pacchetto di discovery ai vari applicativi DDS.
Il secondo attacco che verrà introdotto consente di dirottare il traffico 
del processo di discovery verso un entità a scelta dell'attaccante.
Questo attacco è possibile dato che lo standard OMG non consente 
di effettuare un controllo degli indirizzi IP che devono ricevere 
questo messaggi di tipo discovery.
Infine nell'ultimo attacco che verrà mostrato l'implementazione 
Fast DDS ha una vulnerabilità data la mancanza di controllo di 
un parametro che può essere impostato con lunghezza pari a 0. 
Costruendo un pacchetto RTPS con un i giusti parametri è possibile 
mandare in crash ogni entità della rete che lo riceve.

Nel Capitolo~\ref{chtools}, l'ultimo capitolo di questo elaborato,
verranno presentati degli strumenti di analisi e test del middleware 
che possono essere compatibili anche con altri protocolli o solamente
utilizzabili dal DDS. Il primo tool che verrà introdotto è WireShark, 
un famoso applicativo che si occupa di sniffare i pacchetti scambiati 
all'interno di una rete, nel nostro caso siamo interessati a quelli 
del protocollo RTPS. Infatti successivamente verrà mostrato come 
si suddivide uno di questi pacchetti tra HEADER e sottomessagi in 
modo tale da capire come funziona questo tool. Un'alternativa 
che verrà mostrata è eProsima DDS Record \&
Replay che ha la caratteristica di salvare i pacchetti RTPS salvati 
in un formato specializzato per la lettura di dati serializzati.
Anche il tool che verrà presentato Fast DDS Spy è stato prodotto 
come quello precedente da eProsima e si occupa sempre di analizzare 
il traffico di rete con la caratteristica di riuscire a capire 
quali policy QoS sono attive.
L'ultimo strumento che verrà indicato è DDSFuzz, un applicativo 
per effettuare test di tipo fuzz a al DDS. Prima di questo tool non 
esistevano software simile, quindi dopo la sua creazione sono stati 
scoperte altre vulnerabilità che altri fuzzer più generici non riuscivano 
a trovare dato che non sfruttavano a pieno le caratteristiche del DDS, 
come la possibilità di cambiare la topologia della rete.